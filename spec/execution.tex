%=============================================================================
\section{Implementation}
\label{sec:implementation}

The above section describes a protocol and the format of messages.
Applications can execute, and exchange messages with
gadgets and proving systems implementations in a variety of ways.
We recommend two approaches in this section.

\parhead{In-memory execution}
\label{inmemory}
The application may execute the code of a component in its own process.

The component exposes its functionnality as a function calleable using the C calling convention of the platform. The component code may be linked statically or be loaded from a shared library.

The application must prepare a \textid{ComponentCall} message in memory, and implement one callback functions to receive the messages of the component,
and call the component function with pointers to the message and the callbacks.

The component function reads the call message and performs its specific computation. It prepares the resulting messages of type
\textid{R1CSConstraints}, \textid{AssignedVariables}, or \textid{GadgetReturn}
in memory, and calls the callbacks with pointers to these messages.

A function definition that implements this flow is defined in Listing \ref{gadget.h} as a C header. Refer to the inline documentation for more details.


\parhead{Multi-process execution}

Different parts of the application can be implemented as different
programs, executed separately.

As specified in section~\ref{messagedefinition}, messages are framed and typed,
and can be concatenated in a stream of bytes. It is therefore possible to connect multiple programs through UNIX-style pipes, or to arrange a program to write messages to a file for another program to read later.
To process multiple messages from the same stream or file, a program
reads the 4 bytes containing the size of the next message, allowing it to seek
to the message after that.

A program that constructs a constraint system or implements a gadget should read a GadgetCall message from the standard input stream (stdin).
It should write one or more messages of type
\textid{R1CSConstraints} or \textid{AssignedVariables}
to the standard output stream (stdout).
It should write a single \textid{GadgetReturn} message to the standard error stream
(stderr, used as a control channel).

A program that implements a proving system should read messages of type
\textid{R1CSConstraints}, \textid{AssignedVariables}, or \textid{GadgetReturn}
from stdin, which should contain all the information needed to perform a pre-processing or to generate proofs.


\parhead{Demonstration}

An example implementation is provided.

A number of C++ helper functions are used to interoperate
with libsnark protoboard objects.
A SHA256 gadget from libsnark/gadgetlib1
is encapsulated with the message-based interface of section~\ref{sec:design}.

A test program written in Rust demonstrates
in-memory execution using the method in section~\ref{inmemory},
and includes helper functions to process messages.

This code can be found at \href{https://github.com/QED-it/gadget\_standard}{https://github.com/QED-it/gadget\_standard}.
