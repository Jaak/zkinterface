\section{R1CS Definition}\label{sec:r1cs-def}

\parhead{Background: statement representations}
Many known zk-SNARKs are generic and can express any \emph{NP statement} (i.e., any statement expressed as ``for the instance $x$ there exists a witness $r$ such that relation $D$ holds on $x$ and $w$'' where the relation $D$ is efficiently computable) \cite{ZKProofSecurity}. In principle this already includes convenient high-level representations, and indeed there are research prototypes, for example, for SNARKs supporting general C programs~\cite{BCGTV13csnark-crypto}. However, each zk-SNARK cryptographic construction has a ``native language'' which is a specific type of NP languages. Anything else needs to be converted (reduced) to the native representation, with some overhead. Here we discuss ``R1CS'', one such low-level native representation, and how it can be used to efficiently express useful NP statements to be proven by zk-SNARK constructions.

Moreover, expressing NP statements for SNARKs differs from usual programming (whether in JavaScript or assembly) in the following sense. Usual programming aims to \emph{compute} the output of some function of the given inputs, whereas the NP statement's relation is already given the instance (e.g., the claimed output of a function on some input), and aim to \emph{check} its correctness given some additional information from the witness. This affects the programming style and creates opportunities for important optimizations, such as providing additional advice within the witness that aid the checking.

\parhead{Constraint systems}

Today's best zk-SNARK constructions have a native language consisting of \emph{constraint systems}, i.e., a set of constraints on the instance and witness variables.%
%
\footnote{Colloquially, zk-SNARK developers often refer to constraint systems as ``circuits'', but as we shall soon see, the typical constraint system formalism is more general than Boolean or arithmetic circuits.} %
%
The most common such language is \emph{Rank 1 Constraint System (R1CS)}, also known as \emph{Bilinear Constraint Systems (BCS)}, where each constraint is a (bilinear) quadratic equation over a large finite field.  This is the native language of many efficient zk-SNARKs, such as \cite{GGPR13qsp,PHGR13pinocchio,BCGTV13csnark-crypto,BCTV13von,Groth16size} and many others (see~\cite{ZKProofImplementation}), and is used in practical deployments.
This style of constraint system was introduced in~\cite{GGPR13qsp}.%
%
\footnote{The initial works~\cite{GGPR13qsp,PHGR13pinocchio} used an equivalent formulation called \emph{Quadratic Arithmetic Programs (QAP)}, where the constraints are expressed in terms of polynomials. Later works tend to abstract this away, and do the conversion into polynomials internally.} %
%
The variant defined below is very similar to the definition in \cite[Appendix~E]{BCGTV13csnark-eprint}, used in \href{https://github.com/scipr-lab/libsnark}{libsnark}.%
%
\footnote{To be precise, ``R1CS'', as defined here, is the same as the ``system of rank-1 quadratic equations'' defined in \cite[Appendix~E]{BCGTV13csnark-eprint}, with a minor change in notation. In defining BCS we let $\vwit$ denote just the witness, whereas \cite{BCGTV13csnark-eprint} let $\mathbf{w}$ denote the concatenation of the instance and the witness (and thus added corresponding consistency checks between $\mathbf{w}$ and the instance $\mathbf{x}$). These are also essentially identical to the ``Rank 1 Constraint System (R1CS)'' of \href{https://github.com/scipr-lab/libsnark}{libsnark}.}

An R1CS reasons about two vectors: the \emph{instance} consisting of $\NumInst$ elements and denoted $\vinst=(\inst_1,\ldots,\inst_\NumInst)$, and the \emph{witness} consisting of $\NumWit$ elements and denoted $\vwit=(\wit_1,\ldots,\wit_\NumWit)$. The constraint system says that the two are related by some number of constraints, $\NumConstr$, each of which is a quadratic equation of a specific form. All of the elements and operations are over a large prime field $\Field$, which we will represent here as the integers modulo a large prime $p$.%
\footnote{The specific prime, and the representation of field elements, are related to the elliptic curve used in the QAP-based zkSNARK constructions.}

\begin{definition}
\label{def:bcs}
Let $\NumInst,\NumWit,\NumConstr$ be positive integers, and $\NumConc=\NumInst+\NumWit+1$.
A \defem{Rank 1 Constraint System (R1CS)}, 
over $\Field$ is a tuple
$\System = \big((\vec{\lcoef}_{j},\vec{\rcoef}_{j},\vec{\ocoef}_{j})_{j=1}^{\NumConstr},\NumInst\big)$
where $\vec{\lcoef}_{j},\vec{\rcoef}_{j},\vec{\ocoef}_{j} \in \Field^{\NumConc}$.
Such a system $\System$ is \defem{satisfiable} with an input $\vinst \in \Field^{\NumInst}$ if the following is true:

\NPstatement
  {For these
    $\System = \big((\vec{\lcoef}_{j},\vec{\rcoef}_{j},\vec{\ocoef}_{j})_{j=1}^{\NumConstr},\NumInst\big)$
   and $\vinst \in \Field^{\NumInst}$}
  {there exists a witness $\vwit \in \Field^{\NumWit}$}
  {such that
  $\bigip{\vec{\lcoef}_{j}}{\vconc} \cdot \bigip{\vec{\rcoef}_{j}}{\vconc}  = \bigip{\vec{\ocoef}_{j}}{\vconc}$ \hskip0.4em for all for $j \in [\NumConstr]$, where $\vconc=(1,\vinst,\vwit)\in \Field^{\NumConc}$
  }
\noindent
Above, $\ip{\cdots}{\cdots}$ denotes inner product of vectors over $\Field$, and $\cdot$ denotes multiplication in $\Field$.
\end{definition}

For example, the R1CS
  \begin{equation}\label{eq:bcs-example-formal}
  \System=\left(\left( 
  \begin{aligned}
   \vec{\lcoef}_1 = (0,3,0,0),\, \vec{\rcoef}_1 = (1,1,0,0),\, \vec{\ocoef}_1 = (4,0,5,0) \\
   \vec{\lcoef}_2 = (0,1,2,0),\, \vec{\rcoef}_2 = (6,0,1,0),\, \vec{\ocoef}_2 = (7,0,0,1)
  \end{aligned}
  \right),1 \right)
  \end{equation}
represents the two bilinear constraints
  \begin{equation}
  \begin{aligned}
  \bigip{(0,3,0,0)}{\vconc}\cdot\bigip{(1,1,0,0)}{\vconc} &= \bigip{(4,0,5,0)}{\vconc} \\
  \bigip{(0,1,2,0)}{\vconc}\cdot\bigip{(6,0,1,0)}{\vconc} &= \bigip{(7,0,0,1)}{\vconc} 
  \end{aligned}
  \quad \mbox{ where } \vconc=(1,\inst_1,\wit_1,\wit_2)
  \end{equation}
or simply:
  \begin{equation}\label{eq:bcs-example-informal}
  \begin{aligned}
  3\inst_1 \cdot (1+\inst_1) &= 5\wit_1 + 4 \\
  (\inst_1 + 2\wit_1) \cdot (\wit_1+6) &= \wit_2 + 7  
  \end{aligned}
  \end{equation}
%with $\NumInst=1,\NumWit=2,\NumConstr=2$

By definition, this system $\System$ is satisfiable with input $\inst_1$ if the following is true:
\NPstatementNOperiod
  {For the instance $\inst_1$}
  {there exist a witness $\wit_1,\wit_2$}
  {such that
  \begin{equation*}
  \begin{aligned}
  3\inst_1 \cdot (1+\inst_1) &= 5\wit_1 + 4 \\
  (\inst_1 + 2\wit_1) \cdot (\wit_1+6) &= \wit_2 + 7  
  \end{aligned}
  \end{equation*}
  }

When we prove this NP statement using zk-SNARK, the instance will be the public input that is visible to the verifer, and the witness will be the additional data that is known only to the prover, and used in the construction of the proof, but whose privacy is protected by the zero-knowledge property.

\parhead{Complexity measure}
For backends that consume R1CS natively, the cost of producing the zk-SNARK proofs (in time and memory) typically depends mostly on the \textbf{number of constraints}, $\NumConstr$.

In particular, multiplying variables (or linear combinations thereof) is ``expensive'' (each such multiplication needs a new constraint and thus increases $\NumConstr$), but linear combinations (i.e., additions and multiplications by constants) come ``for free'' in each constraint. 
